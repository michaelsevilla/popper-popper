\section{Conclusion}

We have outlined the structure of our ``reproducible" papers and showed the
complexity of researching in this way. We iterated to a Popper-compliant
paper-writing process after encountering numerous pitfalls, which we have
documented and used to shape our best practices. While our process will never
be perfect, we are encouraged with the improved speed and ease that our research
progresses now that we have: (1) made reproducibility a 1st class citizen, (2)
defined collaboration rules, and (3) made it a priority to maintain pointers.
We hope that our call-to-arms for community cooperation effectively improves
the state of reproducibility in the field.

Our future work is to quantify the efficiency improvements or degradations in
efficiency due to reproducibility. An effective way to quantify productivity,
in addition to the time spent on a problem, would help us more concretely
identify when the reproducibility approach is useful or not. It might also help
us explore other environments where reproducibility is useful, such as the
classroom or in conference settings. Finally, it would highlight situations
where the extra time spent setting up workflows and building paper artifacts is
{\it worth it} for the ultimate payout of reproducibility when the paper is
published.

