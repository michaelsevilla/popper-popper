\section{Popper Pitfalls}
\label{popper-best-practices}

To produce Popper-compliant papers, we needed to change our workflows and
experimental procedures. We have shaped our best practices from the following pitfalls:

\subsection{Reproducibility is not a 1st Class Citizen}
\label{sec:repro}

% exploration vs. complete
Popper-compliance must be observed {\it throughout} the paper-writing process.
Attempts to make published papers Popper-compliant failed for three reasons:
(1) there is no incentive from the perspective of a researcher, (2) it is too
hard to remember the experimental workflow, and (3) the artifacts cannot be
added to the camera-ready article. We started trying to
make~\cite{sevilla:sc15-mantle}
Popper-compliant\footnote{https://github.com/michaelsevilla/mds} but gave up
because the effort would not have been rewarded, as the paper had already been
presented and published.

Making reproducibility a 1st class citizen introduces a careful design
decision: how does the user decide which results are exploratory and which are
complete? We do not want to waste space by saving results that do not
contribute to the paper but, at the same time, how do we know when an
experiment is worthy of being included in the paper? Discipline must be
exercised throughout the paper-writing process to identify useful experiments
and immediately polish them to be Popper-compliant.

% cross-cluster compatibility 
Another pitfall we face is cross-cluster compatibility. After identifying
results as being useful, we usually find cluster-specific files hard-coded
throughout the tree, configuration files randomly propagated to different
directories, visualization files reliant on data specified with hard-coded
paths, and graphs that are not created automatically. To address this, we must
exercise discipline to separate cluster-specific files from cluster-agnostic
files; we do this with \texttt{configs\_*/} directories in our
ceph-popper-template repository.

% organized repos/docs
Finally, organization and documentation must be required throughout the
paper-writing process. We failed to do this in early papers and this causes the
most work later on. System and experiment deploy code is rarely
self-explanatory. At a minimum, each experiment directory must contain a README
specifying how to run the jobs. To address this, we recommend relying heavily
on DockerHub, GitHub, and collaboration; making things public incentivizes 
tidiness and organization. Using internal Docker or Git repositories hampers
Popper-compliance and discourages community involvement.

\subsection{Poorly-Defined Collaboration Roles}

% keeping organization workflows
Collaboration  speeds up code development and, as stated in the previous
section, can be used as motivation for maintaining Popper-compliant repositories;
but collaboration can also hamper the process.  Researchers have different
workflows and preferences. Committing system and experimental deploy code with
different toolkits, documentation, and styles to the same repository can result
in a confusing organizational structure.  To address this, we recommend that
the first author of the paper (1) produce a style guide, either by committing a
couple of experiments or adding detailed READMEs and (2) act as the gatekeeper
for all code. We recommend using GitHub because it nicely formats READMEs and
supports pull requests, issue trackers, and project boards.  Rather than having
meetings to agree on experiment organization, we recommend iteratively
combining code from collaborators with pull requests so the repository grows
organically and quickly in the style approved by the first author.

% merge conflicts
Another obstacle to collaboration is conflicting updates, especially when
editing the paper. We used to assign locks to people in an {\it ad-hoc} fashion
over email but this is not scalable. Instead, we now use Git to manage
conflicts.  Before issuing a pull request, which notifies the first author,
GitHub will notify committers of changes they made in parallel with other
changes. Our policy is to force committers to resolve these merge conflicts
before bothering the first author. While we still recommend issuing pull
requests to make sure the first author approves of the style, the merge
conflict workflow has been an effective way to avoid annoying the first author
with destructive changes.

Git has not totally solved merge conflicts for papers. Papers have long
sentences and sometimes a paragraph is written on a single line of text,
without line breaks. Because Git was designed for code, which has many line
breaks, it is hard to see changes when the document is structured as one long
line. Adding line breaks partially solves the problem but adds new
complexities. For example adding small amounts of text may cause lines to
``spill" onto the next line, so a diff of the commits shows the entire
paragraph as being changed. Similarly, users with different line break
thresholds might re-arrange paragraphs differently, leading to the same
problem. If Git or GitHub were to recognize that these source files were
\LaTeX, the visual problems could be mitigated.

\subsection{Failure to Maintain Pointers}

Popper-compliant papers provide pointers to graph artifacts, including results,
input files, and deploy code. But these links are problematic because they must
be maintained over time, even years after the paper is published. We have had
trouble maintaining these pointers because they (1) are ephemeral, (2) include
pointers to other repositories, and (3) include pointers to different
repository hosting sites. These problems arise because we are using GitHub and
DockerHub as archival tools. Moving artifacts to a proper archival system (with
long term storage and labeling semantics like DOIs) may be an option in the
future, but GitHub works much better for research and exploration. Choosing
between these options is a trade-off we have to make because there is no easy
way for turning an exploration platform like GitHub into an archival system.

%wary of ephemeral links
Ephemeral links and making sure that pointers are live is our biggest pain
point. Small, seemingly benign changes end up confusing users wanting to
reproduce our results. This is an artifact of how the internet was built, but
making sure that these links are live is one of the hardest challenges for
Popper-compliance. To combat this, we suggest maintaining continuous
integration pipelines that read the paper source code and send REST requests to
websites to ensure that 404 codes are not returned.

%large code bases large code bases (src, deploy)
Pointers to other repositories can make the Popper-compliant repository
confusing and difficult to navigate. For example, our Cudele paper had Git
submodule pointers to 3 other GitHub repositories: one for Ansible deployment
code for Ceph, one for Ansible deployment code for monitoring code written by
our research lab, and one for our paper bibliography. Users unfamiliar with
Ansible or \LaTeX would have trouble navigating these large code bases.
Unfortunately, this practice is necessary to avoid duplicating functionality
and ensuring future-proof modules. Our recommendation for addressing this
problem is to minimize the number of repository pointers and to document them
thoroughly, including the reasons that the pointer is necessary and what the
other repository does.

%different registry hubs (Docker, GitHub)
Finally, pointers to different repository sites can make Popper-compliant
papers difficult to understand. Again, this is a necessary evil because
research systems can be large and complicated with many moving parts. For
example, our Cudele paper uses a GitHub repository to maintain our source code
for our modified Ceph version, a GitHub repository to maintain our paper and
deploy code, a GitHub repository to maintain our common deploy code modules
(for monitoring), a DockerHub repository for housing software images with
compiled binaries for our modified Ceph version, and a CloudLab repository to
house our base images. Again, the recommended solution for maintaining this
complicated web of registry hubs is to thoroughly document the process.

%\subsection{Baselines}
% can't always have our own cluster
% continuous integration is hard
% baselines take time 


