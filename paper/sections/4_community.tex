\section{Community Cooperation}
\label{sec:community-cooperation}

We have shown and described reproducibility pitfalls, but equally important is
community buy-in. Next, we outline practices that have been detrimental to our
Popper-compliance initiatives.

\subsection{Conference Requirements}

% double blinded; multiple revision rounds
The most common obstacle to our Popper-compliance efforts is double blinded
submissions. The process is a burden, as we must create anonymous repositories
and remove graph artifacts, but also blocks communication between researchers
and reviewers. Providing evidence that the experiments work gives credence to
the experiment and allows reviewers the chance to examine experiment parameters
more closely. We go a step further and propose removing all anonymity from the
review process, facilitating a communication channel between researchers and
reviewers with the sole intent of improving the quality of the paper. We
applaud the efforts of SC'18, IPDPS'18, and ASPLOS'18 as they move towards this
approach with multiple revision rounds, but argue for more transparent review
processes. We are encouraged that in the review of one of our papers, a reader
specifically asked for source code and reproducibility artifacts; at that
point, we gladly made Popper source links available.

% clear definition of reproducibility, replicability, and open source
A second obstacle is that many conferences lack a clear definition of
reproducibility and replicability, which confuses both submitters and
reviewers. One conference we submitted to had reviewers that posited that
our paper reproducibility artifacts were out of scope while the submission
website clearly had our definition of reproducibility. This confusion is
frustrating and can lead to contentious reviews and rebuttals. To remedy the
situation, we recommend emphasizing reproducibility initiatives to reviewers,
even going as far as to reward papers that have clearly thought about
reproducibility.

\subsection{Industry/Laboratory Requirements}
\label{sec:reqs}

% private/propriety systems --> open architectures
We understand the monetary incentives to propriety systems but working with
code in these environments severely hampers our ability to make papers
Popper-compliant.  Some companies in industry keep all code repositories
private.  Furthermore, many companies have multiple repositories because
development teams like using version control systems ({\it e.g.}, Git or SVN)
and hosting services ({\it e.g.}, GitHub, GitLab, etc.) that they are familiar
with. In larger companies that acquire startups, this is a big problem as every
group of developers brings new ways of managing code.  Obviously, this makes
Popper-compliance impossible, except in a general sense.

We urge companies in industry to adopt a public, unified version control system
and to provide reproducibility artifacts. Obviously, industry entities have
very little incentive to do this. To incentivize this process for industry, we
suggest that conferences provide recognition, either monetary or award-based,
to companies that adopt this philosophy.

% support from upper level management for this process
Another obstacle to Popper-compliance is security. Many systems in national
laboratories require clearance and access is only granted to US citizens. In
fact, many systems are not even connected to the internet to discourage contact
with the outside world. While laboratories are expected to be research havens,
their priorities are obviously security over open research practices. We
recommend evangelizing reproducibility in these communities in the hopes of
appointing ``reproducibility officers" that are internal to the national
laboratories. These officers would have the security clearance to operate large
HPC clusters and the expertise to verify the reproducibility of a paper's
artifacts and performance claims.

